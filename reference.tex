\documentclass[•]{article}
\usepackage{amsmath}
\usepackage{amssymb}
\usepackage{graphicx, subfig}
\usepackage{caption}

\begin{document}

One image ???.
	\begin{figure}[!htbp]
	\centering
	\includegraphics[scale=0.1]{image.jpg}
	\caption{example of one image	} \label{one image}
	\end{figure}

Image together is shown in Figure ???.
First sub-image is shown as Figure ???.
In Figure ??? the second sub-image is presented.
	\begin{figure}[!htbp]
	\centering
	\subfloat[first sub-image]
	{
		\includegraphics[scale=0.1]{image.jpg}
		\label{sub1}
	}
	\qquad
	\subfloat[second image.jpg]

	\subparagraph{second image.jpg}
	\label{sub2}
	
	\caption{combined image}
	\label{img-together}
	\end{figure}

\begin{flushleft}
The result is shown in Equation ???:
\iffalse
\begin{equation}
\label{abcde}
$&a+b+c+d+e=f$
\end{equation}
\fi
\end{flushleft}
	
\end{document}